%\documentclass[a4paper,12pt]{article}
\documentclass{ltxdoc}
\usepackage{multicol}
\usepackage[tt=false, type1=true]{libertine}
\usepackage[varqu,scaled=.88]{zi4}
\usepackage[libertine]{newtxmath}
\usepackage[tableposition=top]{caption}
\usepackage{fancyvrb}
\usepackage{hypdoc}
\usepackage{enumitem}
\setlist%
{%
 topsep=0pt,%
 labelsep=6pt,
 noitemsep,%
 leftmargin=*
}
\setlist[description]{font=\normalfont\sffamily}

\def\file#1{\texttt{#1}}
\begin{document}
\title{\LaTeX\ Class for Elsevier Books}
\author{Deimantas Gal\v{c}ius\footnote{\href{mailto:deimi@vtex.lt}{\texttt{deimi@vtex.lt}}}}
%\date{2017/04/04, v0.1}
\maketitle
\abstract{The package provides a class for typesetting books to be published by Elsevier.}

\tableofcontents

\section{Introduction}

The \file{elsevierbook} class is designed for preparation \LaTeX{} books to be published by Elsevier. 
The document class is built on \file{book.cls} class and requires following packages:

\begin{multicols}{2}
\begin{itemize}
\item \file{etoolbox}
\item \file{calc}
\item \file{afterpackage}
\item \file{numname}
\item \file{xcolor}
\item \file{colortbl}
\item \file{fontenc}
\item \file{textcomp}
\item \file{amsmath}
\item \file{amssymb,amsfonts}
\item \file{amsthm}
\item \file{caption}
\item \file{titlesec}
\item \file{enumitem}
\item \file{mdframed}
\item \file{footmisc}
\item \file{natbib}
\item \file{minitoc}
\item \file{multicol}
\end{itemize}
\end{multicols}

\section{Installation}

%The latest released version of the package can be found on CTAN: \linebreak
%\url{http://www.ctan.org/pkg/elsevierbook/}. 
%The development version can be found on GitHub:
The latest version of the package can be found on GitHub:
 \url{https://github.com/vtex-soft/texsupport. elsevier-book}%, where a bug report can be filed if needed.
. 
%A bug report can be filed at GitHub repository. 
%At this address you can file a bug report—or even
%contribute your own enhancement making a pull request.
The template is prepared using TeXLive distribution  and
is meant to work with the latest TeXLive(MacTeX) \cite{ref:texlive} or MikTeX \cite{ref:miktex} distributions. Please ensure you have up-to-date
LaTeX distribution.



Most users should not attempt to install this package themselves, and rather rely on
their \TeX\ distributions to provide it. If you decide to install the package yourself, follow
the standard rules:
\begin{enumerate}
\item Run latex on \file{elsevierbook.ins}. This will produce the file \file{elsevierbook.cls}.
\item Put the file \file{elsevierbook.cls}  to the places where \LaTeX\ 
can find them (see \cite{ref:ukfaq} or the documentation for your \TeX\ system).
\item Update the database of file names. Again, see \cite{ref:ukfaq} or the documentation for your
\TeX\ system for the system-specific details.
\end{enumerate}


The installation is optional and you can skip this phase.
The bundle is self-contained and after unzipping your have everything you need for a book preparation. 





\section{Book structure}\label{bookstructure}

Book structure may vary depending on whether the book is a monograph or a contributed book. 
Monographs and short books can be organized in simple folder structure as shown in Figure 
\ref{fig:mono}.
Each chapter is prepared in a separate file and the files reside in working directory. 
Image files are put in a separate \file{img/} folder. 
The \LaTeX\ style files have a dedicated \file{sty/} folder.

More complex books such as proceeding volumes or contributed books can have more 
elaborated file structure as shown in Figure \ref{fig:contr}.

\begin{figure}
\begin{minipage}{\dimexpr(\textwidth-0pc)}
\begin{multicols}{2}
\begin{Verbatim}[commandchars=+\(\)]
+fbox(book.tex               )
chapter01.tex
chapter02.tex
appendix01.tex
dedication.tex
preface.tex
bibliography.tex
tableofcontents.tex
[...] 
img/
  ch01-figure01.eps
  ch02-figure02.eps
sty/
  elsevierbook.cls
\end{Verbatim}
\columnbreak
\hrule height0pt
\vskip-2pc
\hrule height0pt
\begin{Verbatim}[frame=single,label={book.tex}]
\csdef{input@path}%
{%
 {sty/},% path for class/sty files
 {img/},% path for graphics files
}
...
\documentclass{elsevierbook.cls}
\begin{document}
\Frontmatter
  %% Title Page %%
\begin{titlepage}%

  \title{Book title}
  \subtitle{Book Subtitle}
  \titlepagerule

  \edition{2nd edition}

  \author{Author Firstname  Surname}
  \address{Author Address}

  \author{Author Firstname  Surname}
  \address{Author Address}

\end{titlepage}

  \begin{dedication}
  %% Dedication page foo can be few\\
  %% line long
\end{dedication}

  \include{tableofcontents}
  \cleardoublepage
\begin{frontmatter}
\chapter*{Preface}% 
\end{frontmatter}

This is an example input file.  Comparing it with
the output it generates can show you how to
produce a simple document of your own.

This is an example input file.  Comparing it with
the output it generates can show you how to
produce a simple document of your own.

This is an example input file.  Comparing it with
the output it generates can show you how to
produce a simple document of your own.

\source%
{%
  \author{Author Name}\\%
  % \address{Location}\\%
  % \date{Month Year}%
}

  \include{acknowledgement}
  [...]
\Mainmatter
  \include{chapter01}
  \include{chapter02}
  [...]
  \appendix
  \include{appendix01}
  [...]
\Backmatter
  \begin{thebibliography}{10}

\bibitem{Arnold74}
V.~I. {Arnol'd}.
\newblock {The asymptotic Hopf invariant and its applications}.
\newblock In {\em Proc. Summer School in Differential Equations}, pages
  133--148, 1974.

\bibitem{Berger84}
M.~A. {Berger}.
\newblock {Rigorous new limits on magnetic helicity dissipation in the solar
  corona}.
\newblock {\em Geophys. Astrophys. Fluid. Dyn.}, 30:79--104, 1984.

\bibitem{Berger99a}
M.~A. {Berger}.
\newblock {Magnetic Helicity in Space Physics}.
\newblock In M.R. {Brown}, R.C. {Canfield}, and A.A. {Pevtsov}, editors, {\em
  Magnetic Helicity in Space and Laboratory Plasmas}, pages 1--9. Geophys.
  Monogr. Ser. 111, AGU, 1999.

\bibitem{Berger03}
M.~A. {Berger}.
\newblock {Topological quantities in magnetohydrodynamics}.
\newblock In A.~{Ferriz-Mas} and M.~{N{\'u}{\~n}ez}, editors, {\em Advances in
  Nonlinear Dynamics}, pages 345--383. Taylor and Francis Group, London, 2003.

\bibitem{BergerF84}
M.~A. {Berger} and G.~B. {Field}.
\newblock {The topological properties of magnetic helicity}.
\newblock {\em J. Fluid. Mech.}, 147:133--148, 1984.

\bibitem{Brandenburg01}
A.~{Brandenburg}.
\newblock {The Inverse Cascade and Nonlinear Alpha-Effect in Simulations of
  Isotropic Helical Hydromagnetic Turbulence}.
\newblock {\em Journal title}, 550:824--840, 2001.

\bibitem{Brown99}
M.R. {Brown}, R.C. {Canfield}, and A.A. {Pevtsov}.
\newblock {\em {Magnetic Helicity in Space and Laboratory Plasmas}}.
\newblock Geophy. Mon. Ser. 111, AGU, 1999.

\bibitem{Chae04}
J.~{Chae}, Y.-J. {Moon}, and Y.-D. {Park}.
\newblock {Determination of magnetic helicity content of solar active regions
  from SOHO/MDI magnetograms}.
\newblock {\em Solar Physics}, 223:39--55, 2004.

\bibitem{Dupont07}
J.-C. {Dupont}, F.~{Schmidt}, and P.~{Koutny}.
\newblock {An example of reference with BibTeX}.
\newblock {\em Solar Physics}, 323:965--985, 2007.

\bibitem{Elsasser56}
W.~M. {Elsasser}.
\newblock {Hydromagnetic Dynamo Theory}.
\newblock {\em Rev. Mod. Phys.}, 28:135--163, 1956.

\bibitem{Heyvaerts84}
J.~{Heyvaerts} and E.~R. {Priest}.
\newblock {Coronal heating by reconnection in DC current systems - A theory
  based on Taylor's hypothesis}.
\newblock {\em Journal title}, 137:63--78, 1984.

\bibitem{Kusano04}
K.~{Kusano}, T.~{Maeshiro}, T.~{Yokoyama}, and T.~{Sakurai}.
\newblock {The Trigger Mechanism of Solar Flares in a Coronal Arcade with
  Reversed Magnetic Shear}.
\newblock {\em Journal title}, 610:537--549, 2004.

\bibitem{Low97}
B.~C. {Low}.
\newblock {The role of coronal mass ejections in solar activity}.
\newblock In N.~{Crooker}, J.~A. {Joselyn}, and J.~{Feynman}, editors, {\em
  Coronal Mass Ejection}, pages 39--48. Geophys. Monogr. Ser. 99, AGU, 1997.

\bibitem{Mandrini05}
C.~H. {Mandrini}, S.~{Pohjolainen}, S.~{Dasso}, L.~M. {Green},
  P.~{D{\'e}moulin}, L.~{van Driel-Gesztelyi}, C.~{Copperwheat}, and
  C.~{Foley}.
\newblock {Interplanetary flux rope ejected from an X-ray bright point. The
  smallest magnetic cloud source-region ever observed}.
\newblock {\em Journal title}, 434:725--740, 2005.

\bibitem{Melrose04}
D.~{Melrose}.
\newblock {Conservation of both current and helicity in a quadrupolar model for
  solar flares}.
\newblock {\em Solar Physics}, 221:121--133, May 2004.

\bibitem{Moffatt69}
H.~K. {Moffatt}.
\newblock {The degree of knottedness of tangled vortex lines}.
\newblock {\em J. Fluid Mech.}, 35:117--129, 1969.

\bibitem{Nindos03}
A.~{Nindos}, J.~{Zhang}, and H.~{Zhang}.
\newblock {The Magnetic Helicity Budget of Solar Active Regions and Coronal
  Mass Ejections}.
\newblock {\em Journal title}, 594:1033--1048, 2003.

\bibitem{Pariat05}
E.~{Pariat}, P.~{D{\'e}moulin}, and M.~A. {Berger}.
\newblock {Photospheric flux density of magnetic helicity}.
\newblock {\em Journal titl}, 439:1191--1203, 2005.

\bibitem{Rust94}
D.~M. {Rust}.
\newblock {Spawning and shedding helical magnetic fields in the solar
  atmosphere}.
\newblock {\em Journal title}, 21:241--244, 1994.
\end{thebibliography}
\end{document}
\end{Verbatim}
\end{multicols}
\end{minipage}
\caption{Folder structure for monographs: 
  folder/file structure (\textit{left}) and contents of 
 the \file{book.tex} file (\textit{right}).}
\label{fig:mono}
\end{figure}


\begin{figure}
\begin{minipage}{\dimexpr(\textwidth-0pc)}
\begin{multicols}{2}
\begin{Verbatim}[commandchars=+\(\)]
book.tex
dedication.tex
preface.tex
references.tex
tableofcontents.tex
[...]
chapter01/
  chapter01.tex
  img/
    ch01-figure01.eps
    ch02-figure02.eps
    [...]
chapter02/
  chapter02.tex
  img/
    ch02-figure01.eps
    ch02-figure02.eps
    [...]
[...]
appendix01/
  appendix01.tex
  img/
    appendix01-figure01.eps
    appendix01-figure02.eps
references.tex
sty/
  elsevierbook.cls
\end{Verbatim}
\columnbreak
\begin{Verbatim}[frame=single,label={book.tex}]
\csdef{input@path}%
{%
 {sty/},% path for class/sty files
 {chapter01/},%
 {chapter01/img/},%
 {chapter02/},%
 {chapter02/img/},%
 {...},%
 {appendix01/},%
 {appendix01/img/},%
}
...
\documentclass{elsevierbook.cls}
\begin{document}
\Frontmatter
  %% Title Page %%
\begin{titlepage}%

  \title{Book title}
  \subtitle{Book Subtitle}
  \titlepagerule

  \edition{2nd edition}

  \author{Author Firstname  Surname}
  \address{Author Address}

  \author{Author Firstname  Surname}
  \address{Author Address}

\end{titlepage}

  \begin{dedication}
  %% Dedication page foo can be few\\
  %% line long
\end{dedication}

  \include{tableofcontents}
  \cleardoublepage
\begin{frontmatter}
\chapter*{Preface}% 
\end{frontmatter}

This is an example input file.  Comparing it with
the output it generates can show you how to
produce a simple document of your own.

This is an example input file.  Comparing it with
the output it generates can show you how to
produce a simple document of your own.

This is an example input file.  Comparing it with
the output it generates can show you how to
produce a simple document of your own.

\source%
{%
  \author{Author Name}\\%
  % \address{Location}\\%
  % \date{Month Year}%
}

  \include{acknowledgement}
  [...]
\Mainmatter
  \include{chapter01}
  \include{chapter02}
  [...]
  \appendix
  \include{appendix01}
  [...]
\Backmatter
  \begin{thebibliography}{10}

\bibitem{Arnold74}
V.~I. {Arnol'd}.
\newblock {The asymptotic Hopf invariant and its applications}.
\newblock In {\em Proc. Summer School in Differential Equations}, pages
  133--148, 1974.

\bibitem{Berger84}
M.~A. {Berger}.
\newblock {Rigorous new limits on magnetic helicity dissipation in the solar
  corona}.
\newblock {\em Geophys. Astrophys. Fluid. Dyn.}, 30:79--104, 1984.

\bibitem{Berger99a}
M.~A. {Berger}.
\newblock {Magnetic Helicity in Space Physics}.
\newblock In M.R. {Brown}, R.C. {Canfield}, and A.A. {Pevtsov}, editors, {\em
  Magnetic Helicity in Space and Laboratory Plasmas}, pages 1--9. Geophys.
  Monogr. Ser. 111, AGU, 1999.

\bibitem{Berger03}
M.~A. {Berger}.
\newblock {Topological quantities in magnetohydrodynamics}.
\newblock In A.~{Ferriz-Mas} and M.~{N{\'u}{\~n}ez}, editors, {\em Advances in
  Nonlinear Dynamics}, pages 345--383. Taylor and Francis Group, London, 2003.

\bibitem{BergerF84}
M.~A. {Berger} and G.~B. {Field}.
\newblock {The topological properties of magnetic helicity}.
\newblock {\em J. Fluid. Mech.}, 147:133--148, 1984.

\bibitem{Brandenburg01}
A.~{Brandenburg}.
\newblock {The Inverse Cascade and Nonlinear Alpha-Effect in Simulations of
  Isotropic Helical Hydromagnetic Turbulence}.
\newblock {\em Journal title}, 550:824--840, 2001.

\bibitem{Brown99}
M.R. {Brown}, R.C. {Canfield}, and A.A. {Pevtsov}.
\newblock {\em {Magnetic Helicity in Space and Laboratory Plasmas}}.
\newblock Geophy. Mon. Ser. 111, AGU, 1999.

\bibitem{Chae04}
J.~{Chae}, Y.-J. {Moon}, and Y.-D. {Park}.
\newblock {Determination of magnetic helicity content of solar active regions
  from SOHO/MDI magnetograms}.
\newblock {\em Solar Physics}, 223:39--55, 2004.

\bibitem{Dupont07}
J.-C. {Dupont}, F.~{Schmidt}, and P.~{Koutny}.
\newblock {An example of reference with BibTeX}.
\newblock {\em Solar Physics}, 323:965--985, 2007.

\bibitem{Elsasser56}
W.~M. {Elsasser}.
\newblock {Hydromagnetic Dynamo Theory}.
\newblock {\em Rev. Mod. Phys.}, 28:135--163, 1956.

\bibitem{Heyvaerts84}
J.~{Heyvaerts} and E.~R. {Priest}.
\newblock {Coronal heating by reconnection in DC current systems - A theory
  based on Taylor's hypothesis}.
\newblock {\em Journal title}, 137:63--78, 1984.

\bibitem{Kusano04}
K.~{Kusano}, T.~{Maeshiro}, T.~{Yokoyama}, and T.~{Sakurai}.
\newblock {The Trigger Mechanism of Solar Flares in a Coronal Arcade with
  Reversed Magnetic Shear}.
\newblock {\em Journal title}, 610:537--549, 2004.

\bibitem{Low97}
B.~C. {Low}.
\newblock {The role of coronal mass ejections in solar activity}.
\newblock In N.~{Crooker}, J.~A. {Joselyn}, and J.~{Feynman}, editors, {\em
  Coronal Mass Ejection}, pages 39--48. Geophys. Monogr. Ser. 99, AGU, 1997.

\bibitem{Mandrini05}
C.~H. {Mandrini}, S.~{Pohjolainen}, S.~{Dasso}, L.~M. {Green},
  P.~{D{\'e}moulin}, L.~{van Driel-Gesztelyi}, C.~{Copperwheat}, and
  C.~{Foley}.
\newblock {Interplanetary flux rope ejected from an X-ray bright point. The
  smallest magnetic cloud source-region ever observed}.
\newblock {\em Journal title}, 434:725--740, 2005.

\bibitem{Melrose04}
D.~{Melrose}.
\newblock {Conservation of both current and helicity in a quadrupolar model for
  solar flares}.
\newblock {\em Solar Physics}, 221:121--133, May 2004.

\bibitem{Moffatt69}
H.~K. {Moffatt}.
\newblock {The degree of knottedness of tangled vortex lines}.
\newblock {\em J. Fluid Mech.}, 35:117--129, 1969.

\bibitem{Nindos03}
A.~{Nindos}, J.~{Zhang}, and H.~{Zhang}.
\newblock {The Magnetic Helicity Budget of Solar Active Regions and Coronal
  Mass Ejections}.
\newblock {\em Journal title}, 594:1033--1048, 2003.

\bibitem{Pariat05}
E.~{Pariat}, P.~{D{\'e}moulin}, and M.~A. {Berger}.
\newblock {Photospheric flux density of magnetic helicity}.
\newblock {\em Journal titl}, 439:1191--1203, 2005.

\bibitem{Rust94}
D.~M. {Rust}.
\newblock {Spawning and shedding helical magnetic fields in the solar
  atmosphere}.
\newblock {\em Journal title}, 21:241--244, 1994.
\end{thebibliography}
\end{document}
\end{Verbatim}
\end{multicols}
\end{minipage}
\caption{Folder structure for contributed books: 
  folder/file structure (\textit{left}) and contents of 
 the \file{book.tex} file (\textit{right}).}
\label{fig:contr}
\end{figure}




\section{Usage}

The class should be loaded with the following command:


\begin{verbatim}
\documentclass[<options>]{elsevierboook}
\end{verbatim}

\noindent where the \verb!options! can be following:

\begin{description}[labelwidth=6pc,labelindent=0pc,labelsep=0pt,leftmargin=6pc]
\item[a02] sets \verb!a02! book model settings
\item[a08a] sets \verb!a08a! book model settings
\item[p05] sets \verb!p05! book model settings
%\item[authoryear] for \file{natbib} package.
\end{description}

Please avoid altering paper dimensions in a \verb!tex! file.


\section{Chapter opener}

All the chapter opening elements are coded inside in a wrapper environment
\verb!\begin{frontmatter}!\ldots\verb!\end{frontmatter}!. A typical 
chapter opener coding is shown below:

\begin{Verbatim}
\begin{frontmatter}
\chapter{Chapter Title\footnote{This is chapter footnote}}%
\subchapter{Chapter Subtitle}
\begin{aug}
\author[addressrefs={ad1,ad2}]%
  {%
    \fnm{Firstname} \snm{Surname}%
    \footnote{This is author footnote}%
  }%
\author[addressrefs={ad2}]%
 {%
   \fnm{Firstname} \snm{Surname}%
 }%
\address[id=ad1]%
  {%
    Name of Institute,
    Division of Department,
    Address of Institute
  }%
\address[id=ad2]%
  {%
    Name of Institute,
    Division of Department,
    Address of Institute
  }%
\end{aug}
\end{Verbatim}

The frontmatter part can have more environments such as abstracts, keywords, chapers points and quotes.
These can be marked up in the following manner:

\begin{Verbatim}
\begin{abstract}
  The ends of words and sentences are marked by spaces. It does not
  matter how many spaces you type; one is as good as 100.  The end of
  a line counts as a space.%

  The ends of words and sentences are marked by spaces. It does not
  matter how many spaces you type; one is as good as 100.  The end of
  a line counts as a space.%
\end{abstract}
\end{Verbatim}

\begin{Verbatim}
\begin{keywords}
  \kwd{key one}
  \kwd{key two}
  \kwd{key three}
\end{keywords}
\end{Verbatim}


\begin{Verbatim}
\begin{chapterpoints}%[Chapter Points]
\item The ends of words and sentences are marked by spaces. It does not
  matter how many spaces you type; one is as good as 100.  The end of
  a line counts as a space.

\item The ends of words and sentences are marked by spaces. It does not
  matter how many spaces you type; one is as good as 100.  The end of
  a line counts as a space.
\end{chapterpoints}
\end{Verbatim}

\begin{Verbatim}
\begin{dispquote}
  The ends of words and sentences are marked by spaces. It does not
  matter how many spaces you type; one is as good as 100.  The end of
  a line counts as a space.

  The ends of words and sentences are marked by spaces. It does not
  matter how many spaces you type; one is as good as 100.  The end of
  a line counts as a space.
  \source{Name}
\end{dispquote}
\end{Verbatim}


\section{Section headings}

There are six section head levels defined. Coding for different heading levels are shown below:
\begin{Verbatim}
\section{Head Level 1}
\subsection{Head Level 2}
\subsubsection{Head Level 3}
\paragraph{Head Level 4}
\subparagraph{Head Level 5}
\subsubparagraph{Head Level 6}
\end{Verbatim}


\section{Lists}

The \file{elsevierbook.cls} class exploits \verb!enumitem! package for list environments 
such as enumerate, itemize and others. It is possible to supply optional arguments 
to fine control the appearance of list (see package documentation for details \cite{ref:enumitem}):
\begin{Verbatim}
\begin{enumerate}[<options>]
\item [...]
\item [...]
\end{enumerate}
\end{Verbatim}


\section{Tables and figures}

Figures may be included using the command \verb!\includegraphics!. 
Use EPS file format for figures  working with LaTeX, and PDF, PNG, MPS file formats for pdfLaTeX. 
Do not use file extensions and path in order to load file. Figure mark up is as follows:
\begin{Verbatim}
\begin{figure}
\includegraphics{file-name}% no path, no extension
\caption{Figure caption \source{Cortesy of [...]}}\label{fig:f01}
\end{figure}
\end{Verbatim}
Table environment may be enhanced depending on the model chosen. 
\begin{Verbatim}
\begin{table}
\begin{tableframe}% frame or background - depends on the model.
\caption{Table caption text [...]  }\label{tbl:t01}
\begin{tabularx}{\textwidth}{X||X}
\tophline
\tch{Item A} & \tch{Item B}\tabnoteref{tn1}\\\hline     % \tch - table column head
\tchi{Item A2} & \tchi{Item B2}\tabnoteref{tn1}\\\hline % \tchi - table column subhead
\rowcolor{thd} \multicolumn{2}{l}{\textbf{Item}} \\     % colored header
Item A & Item B\\\hline
Item C & Item D\tabnoteref{tn2}\\
\bottomhline
\end{tabularx}
\begin{tabnotes}
  \tabnotetext[*]{tn1}{Table footnote}
  \tabnotetext[a]{tn2}{Table footnote}
  \legend{EL=empirical likelihook.}
  \source{foo}
\end{tabnotes}
\end{tableframe}
\end{table}
\end{Verbatim}


\section{Theorems and alike environments}

The class loads \verb!amsthm! package \cite{ref:amsthm} to make it easier 
to define theorem environments and the alike.
\begin{Verbatim}
\newtheorem{theorem}{Theorem}
\theoremstyle{definition}
\newtheorem{definition}{Definition}
\theoremstyle{remark}
\newtheorem{remark}{Remark}
\begin{theorem}[Optional title]\label{thm:01}
...
\end{theorem}
\end{Verbatim}


\section{Boxed text}

The \file{mdframed} package \cite{ref:mdframed} is used for boxed text. 
There are two types of boxed text defined: Box Type A (BtypeA) and 
Box Type B (BtypeB). The mark up for unnumbered boxed text is as following:
\begin{Verbatim}
\begin{textbox}[style=BtypeA,frametitle={Box type A}]
 Some text [...]
\end{textbox}
[...]
\begin{textbox}[style=BtypeB,frametitle={Box type B}]
 Some text [...]
\end{textbox}
\end{Verbatim}
\noindent Numbered boxed text environments are defined the same ways as theorem-like environments:
\begin{Verbatim}
\mdtheorem[style=BtypeA]{example}{Example}[chapter]
\mdtheorem[style=BtypeB]{boxb}{Box}
\begin{example}[Numbered Box type A]
  Some text [..]
\end{example}
[...]
\begin{boxb}[Numbered Box type B]
  Some text [..]
\end{boxb}
\end{Verbatim}



\section{Display mathematics}

AMS math coding is preferred for display mathematics \cite{ref:amsmath}.
Avoid \verb!eqnarray! environment for coding.



\section{Cross-references}

Cross-referencing is possible in \LaTeX\ for section headings, formulae, figure, tables, 
literature references, etc. For example, the words `Fig. 1' will never be more than simple 
text, whereas the proper cross-reference \verb!Fig.~\ref{fig:tiger}! may be turned into a 
hyperlink to the figure. In the same way, the words `Ref. [1]' will fail to turn into a 
hyperlink; the proper cross-reference is \verb!\cite{Knuth96}!.


\section{Bibliography}


The template provided uses the \file{natbib} package for formatting references.
You can choose between author--year (default) and numerical (option \verb!numbers!) citations.
Further customization 
can be made via \verb!\setcitestyle! macro (see \cite{ref:natbib}) for details.
If your bibliography is chapterwise, you may set a \verb!sectionbib! option for \verb~natbib~.
In case of \verb~bibtex~ you may want to use \verb~elsarticle-harv~ or \verb~elsarticle-num~
or your own ~bibtex~ style.
If you are using \verb~biblatex~ for formatting the bibliography, you will probably
need to remove \verb~natbib~ package and load \verb~biblatex~ package.


% \subsection{Bibtex styles}

% Bibtex users can use one of the predefined styles for formatting references (see \cite{ref:elsevier-bibstyles}).

% \begin{description}
% \item[Model 1] Numbered style.
%   \begin{Verbatim}
%     \citestyle{elsarticle-num}
%     \bibliographystyle{../bst/elsarticle-num}
%   \end{Verbatim}
% %
% \item [Model 2] Harvard (name-date) style.
%   \begin{Verbatim}
%     \citestyle{elsarticle-harv}
%     \bibliographystyle{../bst/elsarticle-harv}
%   \end{Verbatim}
% %
% \item [Model 3] Vancouver (numbered) style.
%   \begin{Verbatim}
%     \citestyle{vancouver}
%     \bibliographystyle{../bst/vancouver}
%   \end{Verbatim}
% %
% \item [Model 4] Embellished Vancouver (superscript numbered) style.
%   \begin{Verbatim}
%     \citestyle{vancouver-super}
%     \bibliographystyle{../bst/vancouver-super}
%   \end{Verbatim}
% %
% \item [Model 5] APA (American Psychological Association). Name-date style.
%   \begin{Verbatim}
%     \citestyle{apa}
%     \bibliographystyle{../bst/apa}
%   \end{Verbatim}
% %
% \item [Model 6] AMA (American Medical Association). Numbered style.
%   \begin{Verbatim}
%     \citestyle{ama}
%     \bibliographystyle{../bst/ama}
%   \end{Verbatim}
% %
% \item [Model 7] Saunders (name-date) style.
%   \begin{Verbatim}
%     \citestyle{saunders}
%     \bibliographystyle{../bst/saunders}
%   \end{Verbatim}
% %
% \item [Model 8] Saunders (numbered) style.
%   \begin{Verbatim}
%     \citestyle{saunders-num}
%     \bibliographystyle{../bst/saunders-num}
%   \end{Verbatim}
% %
% \item [Model 9] ACS (numbered) style.
%   \begin{Verbatim}
%     \citestyle{acs-num}
%     \bibliographystyle{../bst/acs-num}
%   \end{Verbatim}
% %
% \item [Model 9a] ACS (superscript numbered) style.
%   \begin{Verbatim}
%     \citestyle{acs-super}
%     \bibliographystyle{../bst/acs-super}
%   \end{Verbatim}
% %
% \item [Model 9b] ACS (name-date) style.
%   \begin{Verbatim}
%     \citestyle{acs}
%     \bibliographystyle{../bst/acs}
%   \end{Verbatim}

% \end{description}


% \subsection{Biblatex}


%\section{Index}

%\section{Acknowledgement}

%\section{Appendices}
%\section{Glossary List}

%\section {Book Frontmatter}

\section{Submission}

Submit one single file as a zip archive. 
Pack your root folder \verb!<your-project-name>! with files and subfolders. 
Check that subfolders \file{sty/}, \file{img/}, or \file{chapterNN/} (if any) are 
present in a zip file.


\section{Bug reports}

Please submit any bug reports, issues or feature requests at \url{https://github.com/vtex-soft/texsupport.elsevier-book/issues}


\begin{thebibliography}{9}
\bibitem{ref:ukfaq} UK \TeX\ Users Group. UK list of \TeX\ frequently asked questions. 
\url{http://www.tex.ac.uk}, 2016.

\bibitem{ref:mdframed} 

M.~Daniel, E.~Schubert. \textit{The mdframed package}, 
2013. \url{http://www.ctan.org/pkg/mdframed}.

\bibitem{ref:amsmath}

F.~Mittelbach, R. Sch\"opf, M. Downes, D.M.~Jones, D.~Carlisle. \textit{AMS mathematical facilities for \LaTeX}, 216. \url{http://www.ctan.org/pkg/amsmath}. 

\bibitem{ref:amsthm}

American Mathematical Society. \textit{Typesetting theorems (AMS style)}, April 2015. 
\url{http://www.ctan.org/pkg/amsthm}.

\bibitem{ref:enumitem}
J. Bezos. 
\textit{Customizing lists with the enumitem package}, 2011. 
\url{http://www.ctan.org/pkg/enumitem}.

\bibitem{ref:natbib}

P.W. Daly.
\textit{Natural Sciences Citations and References}, 
2010. 
\url{https://www.ctan.org/pkg/natbib}.


\bibitem{ref:elsevier-bibstyles}
\textit{Elsevier's Standard Reference Styles}
\url{https://booksite.elsevier.com/9780081019375/content/Elsevier%20Standard%20Reference%20Styles.pdf}

\bibitem{ref:texlive}
  TeX Live \url{texlive.org}.

\bibitem{ref:miktex}
  MikTeX \url{miktex.org}.
  

\end{thebibliography}


\end{document}



%\subsection{titlepage}
%\subsection{Preface}
%\subsection{About Author}
%\subsection{Contributors Lists}
%\subsection{Toc, lof, lot, etc}
%\subsection{Dedication}



